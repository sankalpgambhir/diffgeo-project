
\begin{frame}
    \frametitle{Problem Set 3.1 - 9 --- Karthik Dasigi}
    \emph{Problem Statement}
    A function \(X\rightarrow Y\) between metric spaces is an \textit{isometry} if 
    it preserves distances, that is, \(d(f(x),f(y))=d(x,y)\) for all \(x,y \in X\).
    For instance, the map
    \begin{equation}
        \reals^2 \rightarrow \reals , \text{    } t \mapsto (\frac{3}{5}t+1, \frac{4}{5}t -5)
    \end{equation}
    is an isometry, with the image being the line \(4x =3y + 19\). 
    In which of the metric categories is bijective isometry the notion of isomorphism.
\end{frame}

\begin{frame}
    \frametitle{Isomorphisms}
    \begin{definition}[Isomorphism]
        We say a morphism \(f : a \rightarrow b\) is an isomorphism in \(C\) if there exists a morphism
        \(g : b \rightarrow a\) such that \(f \circ g = id_b\) and \(g \circ f = id_a\). The morphism \(g\) is called the inverse
        of \(f\). The objects \(a\) and \(b\) are said to be isomorphic if there exists an isomorphism \(f:a\rightarrow b\).
    \end{definition}
    \pause
    For two isomorphic metric spaces \(X\) and \(Y\), if \(f:a\rightarrow b\) is an isomorphism and \(g\) its inverse,
    then for any \(x\in X\)
    \begin{equation}
        g(f(x))=id_x(x)=x 
        \label{eqn:gfresult}
    \end{equation}
    
\end{frame}

\begin{frame}
    \frametitle{Isomorphisms in Metric\(_{wc}\)}
    The category Metric\(_{wc}\) is a category whose objects are metric spaces and morphisms are weak contractions.
    \pause
    \begin{definition}[Weak contraction]
        A function \(f : X \rightarrow Y\) between metric spaces is a
        weak contraction if
        \begin{equation}
            d(f(x),f(y))\leq (x,y)
            \label{eqn:wc}
        \end{equation}
    \end{definition}

\end{frame}

\begin{frame}
    Suppose \(f:X\rightarrow Y\) is an isomorphism between \(X,Y\in \text{Metric}_{wc}\), and \(g\) is it's inverse. 
    For \(x_1,x_2\in X\), the isomorphisms \(f\) and \(g\) must satisfy (from \autoref{eqn:wc})
    \begin{gather}
        d(f(x_1),f(x_2))\leq d(x_1,x_2)\\
        d(g(f(x_1)),g(f(x_2)))\leq d(f(x_1),f(x_2))
    \end{gather}

    From \autoref{eqn:gfresult}, we see that 
    \begin{gather*}
        d(f(x_1),f(x_2))\leq d(x_1,x_2)\\
        d(g(f(x_1)),g(f(x_2)))=d(x_1, x_2)\leq d(f(x_1),f(x_2))\\
        \implies d(x_1,x_2)=d(f((x_1),f(x_2)))
    \end{gather*}
    Therefore, isomorphisms in Metric\(_{wc}\) are isometries.

\end{frame}

\begin{frame}
    \frametitle{Isomorphisms in Metric\(_{L}\)}

    

\end{frame}