
\begin{frame}
    \frametitle{Problem Set 3.1 - 9 --- Karthik Dasigi}
    \emph{Problem Statement}

    A function \(X\rightarrow Y\) between metric spaces is an \textit{isometry} if 
    it preserves distances, that is, \(d(f(x),f(y))=d(x,y)\) for all \(x,y \in X\).
    For instance, the map
    \begin{equation}
        \reals \rightarrow \reals^2 , \text{    } t \mapsto (\frac{3}{5}t+1, \frac{4}{5}t -5)
    \end{equation}
    is an isometry, with the image being the line \(4x =3y + 19\). 
    In which of the metric categories is bijective isometry the notion of isomorphism.
\end{frame}

\begin{frame}
    \frametitle{Isomorphisms}
    \begin{definition}[Isomorphism]
        We say a morphism \(f : a \rightarrow b\) is an isomorphism in \(C\) if there exists a morphism
        \(g : b \rightarrow a\) such that \(f \circ g = id_b\) and \(g \circ f = id_a\). The morphism \(g\) is called the inverse
        of \(f\). The objects \(a\) and \(b\) are said to be isomorphic if there exists an isomorphism \(f:a\rightarrow b\).
    \end{definition}
    \pause
    For two isomorphic metric spaces \(X\) and \(Y\), if \(f:a\rightarrow b\) is an isomorphism and \(g\) its inverse,
    then for any \(x\in X\)
    \begin{equation}
        g(f(x))=id_x(x)=x 
        \label{eqn:gfresult}
    \end{equation}
    
\end{frame}

\begin{frame}
    \frametitle{Isomorphisms in Metric\(_{wc}\)}
    The category Metric\(_{wc}\) is a category whose objects are metric spaces and morphisms are weak contractions.
    \pause
    \begin{definition}[Weak contraction]
        A function \(f : X \rightarrow Y\) between metric spaces is a
        weak contraction if
        \begin{equation}
            d(f(x),f(y))\leq d(x,y)
            \label{eqn:wc}
        \end{equation}
        for all \(x,y\in X\)
    \end{definition}

\end{frame}

\begin{frame}
    Suppose \(f:X\rightarrow Y\) is an isomorphism between \(X,Y\in \text{Metric}_{wc}\), and \(g\) is it's inverse. 
    For \(x_1,x_2\in X\), the isomorphisms \(f\) and \(g\) must satisfy the weak contraction property \autoref{eqn:wc}.
    \begin{gather*}
        d(f(x_1),f(x_2))\leq d(x_1,x_2)\\
        d(g(f(x_1)),g(f(x_2)))\leq d(f(x_1),f(x_2))
    \end{gather*}

    Because \(f\) and \(g\) are inverses (\autoref{eqn:gfresult}), we get: 
    \begin{gather*}
        d(g(f(x_1)),g(f(x_2)))=d(x_1, x_2)\leq d(f(x_1),f(x_2))\\
    \end{gather*}
    Using both the inequalities, we get:
    \begin{gather*}
         d(x_1,x_2)=d(f((x_1),f(x_2)))
    \end{gather*}
    \pause
    Therefore, isomorphisms in Metric\(_{wc}\) are bijective isometries.

\end{frame}

\begin{frame}
    \frametitle{Isomorphisms in Metric\(_{L}\)}
    The category Metric\(_{L}\) is a category whose objects are metric spaces and morphisms are Lipschitz continuous maps.
    \pause
    \begin{definition}[Lipschitz continuity]
        A function \(f : X \rightarrow Y\) between metric spaces is
        Lipschitz continuous if there exists a \(K > 0\) such that
        \begin{equation}
            d(f(x),f(y))\leq Kd(x,y)
            \label{eqn:L}
        \end{equation}
        for all \(x,y\in X\)
    \end{definition}    

\end{frame}

\begin{frame}
    Suppose \(f:X\rightarrow Y\) is an isomorphism between \(X,Y\in \text{Metric}_{L}\), and \(g\) is it's inverse. 
    For \(x_1,x_2\in X\), the isomorphisms \(f\) and \(g\) must satisfy (from \autoref{eqn:L})
    \begin{gather*}
        d(f(x_1),f(x_2))\leq Kd(x_1,x_2)\\
        d(g(f(x_1)),g(f(x_2)))\leq K'd(f(x_1),f(x_2))
    \end{gather*}
    Again, by using \autoref{eqn:gfresult} we get
    \begin{gather*}
        d(x_1,x_2)\leq K'd(f(x_1),f(x_2))\\
        \implies KK' \geq 1
    \end{gather*}
    \pause
    This means that for some isomorphisms in Metric\(_{L}\), the coefficients \(K,K'\neq 1\). 
    Thus, not all isomorphisms in Metric\(_L\) are isometries.

\end{frame}

\begin{frame}
    \frametitle{Detour to Problem Set 3.1 - 8}
    In question 8 of the problem set we are asked to prove the following inclusion functors exist:
    \begin{equation*}
        \text{Metric}_{wc} \rightarrow \text{Metric}_L \rightarrow \text{Metric}_u \rightarrow \text{Metric}
    \end{equation*}
    and that the inclusions are not proper.\\

    \pause
    Proofs:
    \begin{itemize}
        \item Proof that weak contractions are Lipschitz continuous:\\
            if \(f:X\rightarrow Y\) is a weak contraction then, 
            \begin{equation*}
                d(f(x_1),f(x_2))\leq d(x_1,x_2)
            \end{equation*}
            thus, \(f:X\rightarrow Y\) is Lipschitz continuous with \(K=1\)
    \end{itemize}
\end{frame}

\begin{frame}
    \begin{itemize}
        \item Proof that Lipschitz continuous functions are uniformly continuous:\\
            if \(f:X\rightarrow Y\) is Lipschitz continuous then,
            \begin{equation*}
                d(f(x_1),f(x_2))\leq Kd(x_1,x_2)\text{ for some }K\geq 0
            \end{equation*}
            if we take \(\delta = \epsilon/K\), for some \(\epsilon>0\), then , 
            \begin{equation*}
                d(x_1,x_2)<\delta \implies d(f(x_1),f(x_2))<\epsilon
            \end{equation*}
            thus, \(f:X\rightarrow Y \) is uniformly continuous
        \pause
        \item Proof that uniform continuous functions are continuous:\\
            if \(f:X\rightarrow Y\) is uniformly continuous then, if given \(\epsilon > 0\), there exists \(\delta>0\) such that:
            \begin{equation*}
                d(x,y)<\delta \implies d(f(x),f(y))<\epsilon
            \end{equation*}
            Clearly, a uniform continuous function is continuous.
    \end{itemize}
\end{frame}

\begin{frame}
    Examples where the inclusions are not proper:
    \begin{itemize}
        \item The function \(f(x)=2x\) on \(\reals\) is Lipschitz continuous but not a weak contraction
        \item The function \(f(x)=\sqrt{x}\) on [\(0,\infty\)] is uniformly continuous but not Lipschitz continuous
        \item The function \(f(x)=1/x\) on [\(0,1\)] is continuous but not uniformly continuous
    \end{itemize}
\end{frame}

\begin{frame}
    \frametitle{Wrapping it up}
    Now that we have shown that we have functors:
    \begin{equation*}
        \text{Metric}_{wc} \rightarrow \text{Metric}_L \rightarrow \text{Metric}_u \rightarrow \text{Metric}
    \end{equation*}
    \pause
    From this we can say that since not all isomorphisms in Metric\(_{L}\) are isometries, 
    not all isomorphisms in Metric\(_u\) and Metric are isometries.\\
    Thus only for the category Metric\(_{wc}\), bijective isometry gives the notion of isomorphism.

\end{frame}