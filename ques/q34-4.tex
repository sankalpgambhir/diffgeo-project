
\begin{frame}
    \frametitle{Question (4)}
    \textbf{Ques - Show that: The underlying topology of the discrete metric is the discrete topology. If a set X has more than one element, then the indiscrete topology on X is not metrizable.}\\\\
    Both these subparts deal with one or the other extreme cases as far as topologies go. So let's look at them individually before solving the problem.\\\\
    \textbf{Discrete Topology:} The textbook definition of a \textit{discrete topology} is that it is a collection of all subsets of $X$, i.e, $\tau = 2^{X}$. There are a few interesting inferences to be drawn from this definition. Since every possible subset is an open subset in the discrete topology, in particular, every \textit{singleton subset} is an open set in this topology.
\end{frame}

\begin{frame}
    \frametitle{Question (4)}
    \textbf{Indiscrete Topology:} The collection $\tau = \{ \emptyset , X \}$ on $X$ is the \textit{indiscrete, or trivial topology} on $X$. A consequence of this collection is that all points in the set $X$ cannot be distinguished from each other through topological means.\\\\
    Now, let's look at the first part of the problem - \\
    \textbf{Show that the underlying topology of the discrete metric is the discrete topology} \\\\
    The discrete metric is as follows - \\
    \[
        d_{\text{discrete}}(x,y) \coloneqq 
        \begin{cases}
            1, & \text{if } x \neq y, \\
            0, & \text{otherwise}.
        \end{cases}
    \]
\end{frame}

\begin{frame}
    \frametitle{Question (4)}
    Now, a metric $d$ on a set $X$ induces a topology $\tau$ by taking the idea of the open balls $B(x,r) = \{y : d(x,y) < r\}$ as basic open sets. We need to show that the $d_{\text{discrete}}$ we are given produces the discrete topology $\tau = 2^{X}$.\\\\
    Let $x \in X$ be an arbitrary element, and let $r \in (0, 1)$; then by the definition of the discrete metric $B_d(x,r) = \{x\}$, so $\forall x \in X, \{x\}$ is an open set.\\\\
    Now, by the axioms we discussed about topological spaces, any arbitrary union of open sets is open. Let $A \subseteq X$ be any arbitrary subset of $X$, then $A = \bigcup_{x \in A} \{x\}$, but we have shown that $\forall x \in X, \{x\}$ is an open set.
\end{frame}

\begin{frame}
    \frametitle{Question (4)}
    Since any arbitrary union of open sets is open, we can claim that $A$ is an open set, as induced by the discrete metric. Since this claim holds for any $A \subseteq X$, we thus claim that every subset of $X$ is open, i.e, $\forall A \subseteq X, A \in \tau$.\\\\
    Since $\tau$ contains every possible subset of $X$, it is the power set $2^X$ of $X$. Thus, we have shown that the discrete metric induces a topology $\tau = 2^X$ on X. Since this is the definition of the discrete topology, we have shown that the underlying topology of the discrete metric is the discrete topology. $\qedsymbol$
\end{frame}

\begin{frame}
    \frametitle{Question (4)}
    We now look at the next part of the problem - \\
    \textbf{Show that if a set $X$ has more than one element, then the indiscrete topology on $X$ is not metrizable.}\\\\
    We prove this by contradiction. Assume that there exists a metric $d$ on the set $X$ such that $(X,d)$ is a metric space and that the topology induced by this metric on $X$ is the indiscrete topology, $\tau = \{ \emptyset , X \}$\\
    $X$ has at least 2 distinct elements $x$ and $y$, i.e, $\exists x,y \in X$ s.t $x \neq y$. \\
    $\implies d(x,y) = r > 0$ \\
    Now, consider the open ball $B(x,r/2)$. This open ball should be an open set in the topology that $d$ induces.
\end{frame}

\begin{frame}
    \frametitle{Question (4)}
    But, $x \in B(x,r/2)$ and since $d(x,y) = r > r/2$, $y \notin B(x,r/2)$. \\\\
    Thus, $B(x,r/2) \neq \emptyset$ and $B(x,r/2) \neq X$ (as there is at least one element $y \in X$ s.t $y \notin B(x,r/2)$).\\\\
    Thus, the topology induced by the metric $d$ cannot be the indiscrete topology, since $\tau_{\text{indiscrete}} = \{\emptyset, X\}$\\\\
    Thus, we have shown that if a set $X$ has more than one element, then the indiscrete topology on $X$ is not metrizable. $\qedsymbol$
\end{frame}