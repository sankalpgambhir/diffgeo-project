\begin{frame}
    \frametitle{Topological Spaces}
    The following questions deal with the idea of Topological Spaces, so here's a quick recap on what exactly those are.\\\\
    \textbf{Topological Spaces:} A \textit{topological space} is a set $X$ on which a \textit{topology} $\tau$ is equipped. $\tau$ is a collection of subsets of $X$ (or, $\tau$ is a subset of the power set $2^{X}$ of $X$) such that -

    \begin{enumerate}
        \item $\varnothing$ and $X$ should belong to $\tau$
        \item the union of the elements in any subset of $\tau$ should belong to $\tau$
        \item the intersection of the elements in any finite subset of $\tau$ should belong to $\tau$
    \end{enumerate}
\end{frame}

\begin{frame}
    \frametitle{Topological Spaces}
    The elements of $\tau$ are called \textit{open sets}. Thus, a topological space is a pair $(X, \tau)$ consisting of a set and a topology on it.\\\\
    We can reframe the axioms given on the previous slide in terms of open sets - 
    \begin{enumerate}
        \item The empty and the full set are open.
        \item Any arbitrary union of open sets is open.
        \item Any finite intersection of open sets is open.
    \end{enumerate}
\end{frame}

\begin{frame}
    \frametitle{Question (8)}
    \textbf{Show that: The euclidean, diamond, square metrics on $\mathbb{R}^2$ have the same underlying topology. (When we say continuous map from $\mathbb{R}^2$ to $\mathbb{R}$, it is w.r.t this topology.) Further, check that it coincides with the product topology on $\mathbb{R} \times \mathbb{R}$.}
\end{frame}