
\begin{frame}
    \frametitle{Problem Set 3.1 - 5 --- Karthik Dasigi}
    \emph{Problem Statement}
    For any metric spaces \(X\) and \(Y\), put three metrics on \(X \times Y\)
    by analogy with the Euclidean, diamon, and square metrics on \(\reals^2\).
    Show that: For any metrix space \(X\), the distance function \(d: X \times X
    \to \reals\) is continuous (wrt either of the three metrics on \(X \times
    X\)).
\end{frame}

\begin{frame}
    \frametitle{Part 1}

    Suppose \(d_x()\) is the distance function on \(X\), and \(d_y()\) is the distance function on \(Y\).
    Analogous to the three metrics on \(\reals^2\), we can create three metrics on \(X\times Y\) that describe the distance between the points \((x_1, y_1)\) and \((x_2, y_2)\):
    \begin{itemize}
        \item Euclidean\(_{X\times Y}\): \(\sqrt{d_x(x_1, x_2)^2 + d_y(y_1, y_2)^2}\)
        \item Square\(_{X\times Y}\): \textbf{max}{\(d_x(x_1, x_2) , d_y(y_1, y_2)\)}
        \item Diamond\(_{X\times Y}\): \(d_x(x_1, x_2) + d_y(y_1, y_2)\)
    \end{itemize}

\end{frame}

\begin{frame}
    \frametitle{Continuity of the distance function}

    We need to now show that the metric on a metric space \(X\), when viewed as
    a function from the larger space \(X\times X\) to \(\reals\) is continuous
    wrt to the metrics defined here. \pause We begin with the case for the
    diamond metric.

\end{frame}

\begin{frame}
    \frametitle{Continuity --- Diamond Metric}

    For the function \(d: X\times X \to \reals\) we need to show that, for each
    \(\epsilon \in \reals > 0\), there exists a \(\delta\) such that 

    \begin{gather}
        \forall (x, y), (x_0, y_0) \in X\times X,\nonumber\\
        \Delta((x, y), (x_0, y_0)) < \delta \Rightarrow |\metric{x}{y} - \metric{x_0}{y_0}| < \epsilon~.
    \end{gather}

    Note that \(\Delta((x, y), (x_0, y_0)) = \metric{x}{x_0} + \metric{y}{y_0}\)
    in the diamond formalism.

    % note that the left side of the implication is simply the distance function
    % over X x X, and the right side is the same over R
    
\end{frame}

\begin{frame}
    We begin with the given bound \(\epsilon\) such that

    \begin{gather}
        |\metric{x}{y} - \metric{x_0}{y_0}| < \epsilon~, \nonumber\\
        \metric{x}{y} - \metric{x_0}{y_0} < \epsilon~, \nonumber\\
        \metric{x}{y} +\metric{x_0}{y_0}< \epsilon + 2\metric{x_0}{y_0}~.
    \end{gather}

    \pause

    Adding \(2\metric{y}{x_0}\) to both sides and using the triangle inequality,
    we obtain

    \begin{tcolorbox}[colframe=red, colback=red!10!white]
        \begin{align}
            T = \metric{x}{x_0} + \metric{y}{y_0} &\leq (\metric{x}{y} + \metric{y}{x_0}) + (\metric{y}{x_0}+\metric{x_0}{y_0}) \nonumber\\ 
            &< \epsilon + 2\metric{x_0}{y_0} + 2\metric{y}{x_0}~.
            \label{eqn:tresult}
        \end{align}
    \end{tcolorbox}

    % \pause
    %
    % And similarly by adding \(\metric{x}{y_0}\), and finally adding the two
    % resultant inequalities, we get

    % \begin{gather}
    %     \metric{y}{y_0} \leq \metric{x}{y} + \metric{x}{y_0} < \epsilon + \metric{x_0}{y_0} + \metric{x}{y_0}~,\nonumber\\
    %     \textcolor{red}{T = \metric{x}{x_0} + \metric{y}{y_0} < 2\epsilon + 2\metric{x_0}{y_0} + \metric{y}{x_0}+ \metric{x}{y_0}~.}        
    %     \label{eqn:tresult}
    % \end{gather}

    % comment on T being the required quantity now we need to resolve the 1D
    % terms on the right side, since x and y are variable

\end{frame}

\begin{frame}
    \frametitle{Peeking along a side}

    To resolve the terms dependant on only \(y\), let us consider
    the dimensionally reduced problem, or proving continuity along the y-axis
    inside the \(\epsilon\)-ball around the point \((x_0, y_0)\), i.e.

    \begin{gather*}
        \text{Given } \epsilon \in \reals > 0 \text{ as before, with} \\
        |\metric{x_0}{y} - \metric{x_0}{y_0}| < \epsilon~.
    \end{gather*}

    \pause

    It is easy to now see the reduction

    \begin{gather}
        \metric{x_0}{y} - \metric{x_0}{y_0} < \epsilon~, \nonumber \\
        \metric{x_0}{y} < \epsilon + \metric{x_0}{y_0}~.
        \label{eqn:ybound}
    \end{gather}

    We see that this is a bound on the \(y\) dependant term that is constant
    (freezing the reference point).

\end{frame}

\begin{frame}
    \frametitle{Collecting results}

    From the results for \(T \text{ and } y\) in \autoref{eqn:tresult} and
    \autoref{eqn:ybound}, we obtain the bound on \(T\)

    \begin{equation}
        T < \epsilon + 2\metric{x_0}{y_0} + 2\metric{y}{x_0} < 3\epsilon + 4\metric{x_0}{y_0} = \delta(\epsilon)
    \end{equation}

    as required.

\end{frame}

% write about the proof for other metrics as well