
\begin{frame}
    \frametitle{Problem Set 3.1 - 5 --- Karthik Dasigi}
    \emph{Problem Statement}

    For any metric spaces \(X\) and \(Y\), put three metrics on \(X \times Y\)
    by analogy with the Euclidean, diamond, and square metrics on \(\reals^2\).
    Show that: For any metrix space \(X\), the distance function \(d: X \times X
    \to \reals\) is continuous (wrt either of the three metrics on \(X \times
    X\)).
\end{frame}

\begin{frame}
    \frametitle{Metric analogues}

    Suppose \(d_x\) is the distance function on \(X\), and \(d_y\) is the distance function on \(Y\).
    Analogous to the three metrics on \(\reals^2\), we can create three metrics on \(X\times Y\) that
    describe the distance between the points \((x_1, y_1)\) and \((x_2, y_2)\):
    \begin{itemize}
        \item Euclidean\(_{X\times Y}\): \(\sqrt{d_x(x_1, x_2)^2 + d_y(y_1, y_2)^2}\)
        \item Square\(_{X\times Y}\): \textbf{max}(\(d_x(x_1, x_2) , d_y(y_1, y_2)\))
        \item Diamond\(_{X\times Y}\): \(d_x(x_1, x_2) + d_y(y_1, y_2)\)
    \end{itemize}

\end{frame}

\begin{frame}
    \frametitle{Continuity of the distance function}

    We need to now show that the metric on a metric space \(X\), when viewed as
    a function from the larger space \(X\times X\) to \(\reals\) is continuous
    wrt to the metrics defined here. 

\end{frame}

\begin{frame}

    \frametitle{Continuity}

    \begin{definition}[Continuity]
        Suppose \(X\) and \(Y\) are metric spaces. A function \(f : X \rightarrow Y\)
        is continuous if for any point \(x_0 \in X\), given \(\epsilon > 0\),
        there exists \(\delta > 0\) such that
        \begin{equation}
            d(x,x_0)<\delta \text{ implies } d(f(x), f(x_0))<\epsilon 
        \end{equation}

    \end{definition}
    
\end{frame}

\begin{frame}
    \frametitle{Continuity --- Diamond metric}

    We begin with the case for the
    diamond metric.

    \pause

    For the function \(d: X\times X \to \reals\) we need to show that, for some point \((x_0, y_0)\ \in X\times X \),
    given an \(\epsilon > 0\), there exists a \(\delta\) such that 

    \begin{gather}
        \Delta((x, y), (x_0, y_0)) < \delta \Rightarrow |\metric{x}{y} - \metric{x_0}{y_0}| < \epsilon~.
    \end{gather}

    Note that the diamond metric \(\Delta((x, y), (x_0, y_0)) = \metric{x}{x_0} + \metric{y}{y_0}\)
    and the function \(d\) is the metric on \(X\).

    % note that the left side of the implication is simply the distance function
    % over X x X, and the right side is the same over R
    
\end{frame}

\begin{frame}
    We begin with the bound \(\delta(\epsilon)\) on \(\Delta((x, y), (x_0, y_0)\) 
    \begin{gather*}
        \metric{x}{x_0} - \metric{y}{y_0} < \delta(\epsilon)
    \end{gather*}
    \pause
    Using the triangle inequality on \(\metric{x}{x_0}\) we get the following:
    \begin{gather*}
        \metric{x}{x_0} \geq \metric{x}{y} - \metric{x_0}{y} \\
        \text{and,}\\ \metric{x}{x_0} \geq \metric{x_0}{y} - \metric{x}{y} \\
        \Rightarrow \metric{x}{x_0} \geq |\metric{x}{y} - \metric{x_0}{y}| 
    \end{gather*}
    \pause
    Similarly,
    \begin{gather*}
        \metric{y}{y_0} \geq |\metric{x_0}{y} - \metric{x_0}{y_0}|
    \end{gather*}
    \pause
    Thus,
    \begin{equation*}
        |\metric{x}{y} - \metric{x_0}{y}| + |\metric{x_0}{y} - \metric{x_0}{y_0}| \leq \metric{x}{x_0} - \metric{y}{y_0} < \delta(\epsilon)
    \end{equation*}
\end{frame}

\begin{frame}
    Now, employing the inequality \(|a+b|\leq |a|+|b|\), we get:
    \begin{equation*}
        |\metric{x}{y} - \metric{x_0}{y_0}| < \delta(\epsilon) \\
        \Rightarrow \delta(\epsilon) = \epsilon
    \end{equation*}

    Thus we have shown that for a given \(\epsilon\), there exists a \(\delta (= \epsilon\)) such that 
    \begin{equation*}
        \Delta((x, y), (x_0, y_0)) < \delta \Rightarrow |\metric{x}{y} - \metric{x_0}{y_0}| < \epsilon
    \end{equation*}

    And hence, the distance function \(d:X\times X \rightarrow \reals \) is continuous.

\end{frame}


\begin{frame}
    \frametitle{Continuity for other metrics}

    To prove the continuity for the Euclidean and Square metric, we can expand on the proof covered in the previous slides.

    \pause
    To expand the proof, we make use of the following property:
    for positive \(x\) and \(y\),
     
    \begin{equation}
        x+y \geq (x^p+y^p)^{1/p} \text{ for } p \geq 1
    \end{equation}

    (This is can be viewed as the \textit{Minkowski} inequality applied to a 1-dimensional vector)

    \pause

    This means that the same \(\delta(\epsilon)\) can be used for the proof of continuity of the other two metrics.
    \begin{equation}
        (\metric{x}{x_0}^p + \metric{y}{y_0}^p)^{1/p} \leq \metric{x}{x_0} + \metric{y}{y_0} < 3\epsilon + 4\metric{x_0}{y_0} = \delta(\epsilon)
    \end{equation}

    Putting \(p = 2\) proves continuity for the Euclidean metric, and \(p=\infty\) proves continuity for the Square metric
    
\end{frame}

% write about the proof for other metrics as well