
\begin{frame}
    \frametitle{Problem Set 3.4 - 7 --- Pushkar Mohile}
    Analysis Notes 3.4 Q.- Show that The discrete and indiscrete topologies on a
    set give rise to functors

\begin{equation}
    \cat{Set}\to \cat{Top}
\end{equation}    
    
    and these are the left and right adjoints, respectively, to the forgetful
    functor from Top to Set. 
\end{frame}
\begin{frame}
    
Solution : 

We begin by recalling the definition of a functor. Given two Categories
\(\textsf{C}\) amd \(\textsf{D}\), A functor \(\mathcal{F} \) assigns to every
object of a \(\textsf{C}\) an object \(\mathcal{F}(a)\) in \(\textsf{D}\) \\
For every morphism \(f \in C(a,b)\) a corresponding morphism 
\begin{equation}
\mathcal{F}(f) \in \text{D}(\mathcal{F}(a), \mathcal{F}(b) )
\end{equation}
which respects composition 
\begin{equation}
    \mathcal{F}(f\circ g) = \mathcal{F}(f) \circ \mathcal{F}(g)
\end{equation}
    and maps \(\textsf{id} \text{ to } \textsf{id}\) 
\end{frame}

\begin{frame}
    
Let us construct the functors corresponding to the discrete and indiscrete
topologies on any set \(X\) given by \(\tau_{disc} = 2^X\) and \(\tau_{indisc} =
\{\phi,X \}\). We will call them \textit{disc}: \(\textsf{Set} \to
\textsf{Top}\) and \textit{indisc}: \(\textsf{Set}\to \textsf{Top}\) , defined
in the following way 

\begin{gather*}
    \textit{disc}(X) \mapsto (X, \tau_{disc}) \\
    \textit{indisc}(X) \mapsto (X,\tau_{indisc}) 
\end{gather*}
And for any function \(f \in \textsf{Set}(X,Y)\), \(f\mapsto f\). 
% Rewrite this later . Check Swapneel's way of writing this and write in the
% same style. 
The check we need to make here is that \(f\) is a continous function between
sets with the discrete and indiscrete topology. 
\end{frame}

\begin{frame}
    
 For the discrete topology, this is done by noting that 
 
 \begin{equation}
    \forall U \in \tau_Y , f^-1 (U) \subseteq X \in 2^X
 \end{equation}

    and hence \(f^{-1}(U)\) is open in \(\tau_{disc}\). 
    
  Similarly, for the indiscrete topology, the only open subsets of \(Y\) are
    \({\phi,Y}\).  \\ For \(\phi \in \tau_Y\) we have \(f^{-1}(\phi) = \phi \in
    \tau_X \) and \(f^{-1}(Y) = X \cup \phi \in \tau_X \)  %\note{Clarify}
and hence once again \(f\) is continous. The composition law is valid since
composition of continous functions are continous. Hence the discrete and
indiscrete topologies define the required functors. 

\end{frame}

\begin{frame}
    
For the second part, we have to show that these are left adjoint and right
adjoints respectively to the forgetful functor defined as follows: 
\begin{gather*}
    frg : \textsf{Top} \to \textsf{Set} \\
    (X, \tau_X ) \mapsto X \\
    f \in \textsf{Top}((X,\tau_X),(Y,tau_Y ) \mapsto f \in \textsf{Set}(X,Y) 
\end{gather*}
ie we are forgetting the underlying topology and viewing the function \(f\) as a
morphism between sets. 
\end{frame}
\begin{frame}
    
We recall the definitions of left and right adjoint functors. Given two
categories \(\textsf{C}\) and \(\textsf{D}\) and functors \(\mathcal{F}:
\textsf{C} \to \textsf{D}\) and \(\mathcal{G}: \textsf{D} to \textsf{C}\), we
say that \(\mathcal{F}\) is a left adjoint and \(\mathcal{G}\) is a right
adjoint if for objects \(a\) in \textsf{C} and \(x\) in \(\textsf{G}\) , there
is a bijection between the set of morphisms 

\(\textsf{D}(\mathcal{F}(a) , x) \xrightarrow{\cong} \textsf{C}(a,
\mathcal{G}(b))\)

that is \textit{natural} in \(a\) and \(x\). The naturality condition is
formally stated as follows: For any morphism \(a \to a'\) in \textsf{C} and \(x
\to x'\) in \textsf{D}, we have the following commutative diagrams (Check Cat
Theory lec. 2 or section 5 of the notes )
%insert comm diagram here Insert comment about what naturality axiom is telling
%us 
\end{frame}
\begin{frame}
    
We now check the adjuction between \(frg\) and \(disc\) as  defined previously.
Let \(X \) be any set and \((Y,\tau_Y)\) be any topological space. We have to
prove that 
\begin{equation}
    \cat{Top}(\textit{disc}(X) , (Y, \tau_Y)  ) \bijec \cat{Set}(X, Y)
\end{equation}
This bijection is given by  \(f \mapsto f \) in both directions. We now have to
simply check whether the two sets are the same. This is done as follows: 
\begin{equation}
    \cat{Top}(\textit{disc}(X) , (Y, \tau_Y)  ) \subset \bijec \cat{Set}(X, Y)
\end{equation}
is obvious since continous functions are functions between the sets. 
\end{frame}
\begin{frame}
    
Next, note that 
\begin{equation}
    \cat{Set}(X, Y) \subset \cat{Top}(\textit{disc}(X) , (Y, \tau_Y)  )
\end{equation}
Proof: Let \(f \in \cat{Set}(X,Y)\),  \(U_Y\) be any open set on \(\tau_Y\).
\(f^{-1}(U) \subseteq X \in 2^X\) and hence \(f\) is continous. \\
Thus we have proved that the two sets are equal. Finally we make note of the
naturality condition. This holds because composition of functions and
composition of continuos function commute with the functors. 
%Draw a diagram. 
\end{frame}
\begin{frame}
    
This adjunction can be restated in terms of the following universal property of
the discrete topology : The discrete topology on \(X\) is the topology such that
every function from \(X\) to any topological space \(Y, \tau_Y\) is continuos. 
%Draw the commutative diagram 
\end{frame}

\begin{frame}
    Finally we take a look at the right adjoint condition for the indiscrete
    topology. The conditions states that 
    \begin{equation}
         \cat{Set}(Y,X ) \bijec \cat{Top}((Y, \tau_Y) ,\textit{indisc}(X))
    \end{equation}
    The checks are similar to the previously done checks.  We mention the only
    nontrivial check: 
    \begin{equation}
        \cat{Set}(Y,X ) \subseteq\cat{Top}((Y, \tau_Y) ,\textit{indisc}(X))
    \end{equation}
    For a given function \(f \in   \cat{Set}(Y,X )\), with the indiscrete
    topology on \(X\), \(f^{-1}(\phi) = \phi\) and \(f^{-1}(X) = Y \cup \phi\),
    both of which are open wrt any topology on \(Y\). Hence \(f\) is continous. 

\end{frame}
